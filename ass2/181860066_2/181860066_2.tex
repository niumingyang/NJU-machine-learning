\documentclass{article}
\usepackage{blindtext}
\usepackage[utf8]{inputenc}
\usepackage{amsmath,bm}
\usepackage{amstext}
\usepackage{amsfonts}
\usepackage{amsmath}
\usepackage[colorlinks,linkcolor=blue]{hyperref}
\usepackage{CTEX}


\title{Introduction to Machine Learning\\Homework 2}
\author{181860066 牛铭杨}
\begin{document}
	\maketitle
	\numberwithin{equation}{section}
	\section{[30pts] Multi-Label Logistic Regression}
    In multi-label problem, each instance $\bm{x}$ has a label set $\bm{y}=\{y_1,y_2,...,y_L\}$ and each label $y_i\in\{0,1\}, \forall 1 \leq i \leq L$. Assume the post probability $p(\bm{y} \mid \bm{x})$ follows the conditional independence:\\
    \begin{equation}
    p(\bm{y} \mid \bm{x})=\prod\limits_{i=1}^L p(y_i \mid \bm{x}).
    \end{equation}
    Please use the logistic regression method to handle the following questions.\\
    (1) [15pts] Please give the log-likelihood function of your logistic regression model;\\
        给定数据集$\left\{\bm{x_i}, \bm{y_i}\right\}_{i=1}^m$,对数似然函数为
        \begin{equation}
            l({\bm w}, b) = \sum_{i=1}^m {\rm ln} p({\bm y_i}|{\bm x_i};{\bm w},b)
                         = \sum_{j=1}^L \sum_{i=1}^m {\rm ln} p({y_{ij}}|{\bm x_i};{\bm w},b)
        \end{equation}
        其中${\bm y_{ij}}$是第$i$个样本在第$j$个标签上的属性\\
        所以采用书上的记号,只需最小化
        \begin{equation}
            l({\bm \beta}) = \sum_{i=1}^m (-\sum_{j=1}^L y_{ij} {\bm \beta^T} {{\bm {\widehat x}_i}}
            +L{\rm ln}(1+e^{{\bm \beta^T} {{\bm {\widehat x}_i}}}))
        \end{equation}\\
    
    (2) [15pts] Please calculate the gradient of your log-likelihood function and show the parameters updating step using gradient descent.\\
    似然函数的梯度为
    \begin{equation}
        \nabla l({\bm \beta}) = \sum_{i=1}^m {{\bm {\widehat x}_i}} (-\sum_{j=1}^L y_{ij}
        +\frac{Le^{{\bm \beta^T}{{\bm {\widehat x}_i}}}}{1+e^{{\bm \beta^T}{{\bm {\widehat x}_i}}}})
    \end{equation}
    第$t+1$轮迭代解的更新公式为
    \begin{equation}
        \begin{aligned}
            {\bm \beta}^{t+1}&={\bm \beta}^t-\gamma \nabla l({\bm \beta})
            &={\bm \beta}^t-\gamma \sum_{i=1}^m {{\bm {\widehat x}_i}} (-\sum_{j=1}^L y_{ij}
            +\frac{Le^{{\bm \beta^T}{{\bm {\widehat x}_i}}}}{1+e^{{\bm \beta^T}{{\bm {\widehat x}_i}}}})
        \end{aligned}
    \end{equation}
    


\numberwithin{equation}{section}
\section{[70pts] Logistic Regression from scratch  }
\subsection{实现细节}
    我将这次实验分为3个阶段,导入训练数据,梯度下降/上升法训练出${\bm w} = ({\bm w}; b)$,根据测试集预测结果。

    导入数据时使用两个矩阵$X,Y$来存储输入的样本和样本的标签,其中$X$每个行向量(样本)都加上一列,其值为1,以便计算。

    然后使用$OvR$,训练$26$个分类器,在训练每个分类器时,作$Y$的深拷贝,并将当前识别的那类标注为正例,其他均为反例。
    
    使用梯度下降/上升法训练出每个分类器的${\bm w}$,测试时识别为概率最高的那一类。测试时,可以统计每个类别的混淆矩阵,之后总计即可算出查全率和查准率以及$F1$。

    梯度下降/上升法每一步沿着梯度的方向前进一个步长的距离,经过多次迭代就可以近似最值。我们要最大化
\begin{equation}
    l({\bm w}) = \sum_{i=1}^m (y_{i} {\bm w^T} {{\bm {\widehat x}_i}}
            -{\rm ln}(1+e^{{\bm w^T} {{\bm {\widehat x}_i}}}))
\end{equation}
对其求偏导,得到
\begin{equation}
    \begin{aligned}
        \frac{\partial l({\bm w})}{\partial w_j} 
        &= \sum_{i=1}^m (y_{i}  x_{ij}
            -x_{ij} \frac{e^{{\bm w^T} {{\bm {\widehat x}_i}}}}{1+e^{{\bm w^T} {{\bm {\widehat x}_i}}}})\\
        &= \sum_{i=1}^m x_{ij}(y_{i}  
        - \frac{e^{{\bm w^T} {{\bm {\widehat x}_i}}}}{1+e^{{\bm w^T} {{\bm {\widehat x}_i}}}})\\
        &= \sum_{i=1}^m x_{ij}(y_{i} - Sigmoid({\bm w^T} {{\bm {\widehat x}_i}}))\\
    \end{aligned}
\end{equation}
所以有
\begin{equation}
    \begin{aligned}
        \nabla l({\bm w}) = X^T(Y - Sigmoid(X {\bm w} ))
    \end{aligned}
\end{equation}
而第$t+1$轮迭代解的更新公式为
\begin{equation}
    {\bm w}_{t+1} = {\bm w}_t + \gamma \nabla l({\bm w}) = {\bm w}_t + \gamma X^T(Y - Sigmoid(X {\bm w} ))\label{mylabel2}
\end{equation}
其中$\gamma$为步长,上面(\ref{mylabel2})即为我的实现参照的等式

\subsection{优化与参数设置}
实验的步长参数$\gamma$和循环次数$loops$是可以调节的,我通过调节这两个参数来使实验效果更好。

一开始我设置$\gamma = 0.001, loops = 1000$,效果并不好,准确率只能达到百分四十几,
然后我意识到可能并不需要循环这么多次,就已经收敛了,步长太大每次都越过了最值点。
所以我根据循环次数来调节步长,经过不断实践,我发现,$loops < 400$时,$\gamma = 0.001$,
$loops < 480$时,$\gamma = 0.0001$,$loops < 500$时,$\gamma = 0.00001$,
这样是较好的选择,准确率可以达到百分之六十几,而仅仅增加循环次数效果并不是很理想。

下面是实现的结果。

\begin{table}[h]
    \centering
     \caption{Performance on test set.}
     \vspace{2mm}
    \label{tab:my_label}
    \begin{tabular}{|c|c|}
       \hline
       Performance Metric & Value (\%) \\
       \hline
       accuracy & 66.95 \\
       \hline
       micro Precision  & 65.22\\
       \hline
       micro Recall & 66.15\\
       \hline
       micro $F_1$ & 65.68\\
       \hline
       macro Precision  & 65.23\\
       \hline
       macro Recall & 68.55\\
       \hline
       macro $F_1$ & 66.85\\
       \hline
    \end{tabular}

\end{table}
\end{document}